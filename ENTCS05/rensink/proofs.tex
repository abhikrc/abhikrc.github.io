\clearpage
\section{Proofs of the theorems}
\alabel{proofs}

\begin{oldresult}{Proposition}{p:pre-shaping}
  Let $L,G\in \DetGraph_\L$ and $S\in \Shape_\L$. If $f\ftype L G$ is an
  injective morphism and $s\ftype G S$ a shaping, then $s\circ f$ is a
  pre-shaping of $L$ into $S$.
\end{oldresult}
%
\begin{proof}
  Let $p=s\circ f$.  Clearly $p$ is a graph morphism; we only have to show
  satisfaction of the multiplicities' upper bounds, in the sense of
  \dref{pre-shaping}.
%
\begin{itemize}\smalltopsep
\item Let $v\in N_S$ be arbitrary. It follows (by the fact that $s$ is a
  shaping) that $s^{-1}(v):\nodeMult_S(v)$, which implies (among other things)
  $\card{s^{-1}(v)}\leq \ubnd{\nodeMult_S(v)}$. From the injectivity of $f$ it
  follows that $\card{f^{-1}(s^{-1}(v))} \leq \card{s^{-1}(v)}$, hence we are
  done.
  
\item Let $v\in N_L$ and $a\in \L$ be arbitrary. Since $s$ is a
  shaping it follows (among other things) that $\card{\genset{w\in
  N_G}{(w,a,f(v)) \in E_G}} \leq \ubnd{\inMult_S(s(v))(a)}$. From the
  injectivity of $f$ it follows that 
  \[ \card{\genset{w\in N_L}{(f(w),a,f(v)) \in E_G}} \leq
     \card{\genset{w\in N_G}{(w,a,f(v)) \in E_G}} \enspace;
  \]
  moreover, since $f$ is a graph morphism we have $\genset{w\in N_L}{(w,a,v)
  \in E_L} \subseteq \genset{w\in N_L}{(f(w),a,f(v)) \in E_G}$. These three
  inequalities suffice to conclude the proof obligation.\qed
\end{itemize}
\end{proof}
%
\begin{oldresult}{Proposition}{p:concrete pre-shaping}
  Let $L\in \DetGraph_\L$ and $S\in \Shape_\L$ and let $c\ftype L S$ be a
  concrete pre-shaping. For any $G\in \DetGraph_\L$ with a shaping $s\ftype G
  S$, there is an injective morphism $m\ftype L G$ such that $c=s\circ m$.
\end{oldresult}
%
\begin{proof}
By the fact that $c$ is concrete, it follows that
$\nodeMult_S(c(v))= \meq1$ for all $v\in N_L$; hence $s(w)=c(v)$ uniquely
identifies $w\in N_G$, and so $s^{-1}(c(v))$ is well-defined. We make use of
this fact by defining a node mapping $m\ftype{N_L}{N_G}$ as
%
\[ m : v \mapsto s^{-1}(c(v)) \enspace. \]
%
Since $c=s\circ m$ holds by construction, we only have to show that $m$ is an
injective graph morphism. 

Let $(v,a,w)\in E_L$ be arbitrary. It follows that $(c(v),a,c(w))\in E_S$; but
then (due to Clause~3 of \dref{shaping}) $\exists(m(v),a,w')\in N_G$, and so
$(c(v),a,s(w'))\in E_S$. Because $c$ is concrete, it follows that $s(w')=c(w)$,
and so $w'=m(w)$. We may conclude that $m$ is indeed a graph morphism.

The injectivity of $m$ follows from the fact that $c$ is injective (which is
enforced by the node multiplicity $\nodeMult_S(c(v))= \meq1$ for all $v\in
N_L$). \qed
\end{proof}
%
\begin{oldresult}{Proposition}{p:mat construction}
  Let $L\in \DetGraph_\L$ and $S\in \Shape_\L$, and let $p\ftype L S$ be a
  pre-shaping. $\alpha_p$ gives rise to an abstraction morphism from $\mat p S$
  to $S$, and $\id_L$ gives rise to a concrete pre-shaping of $L$ into $\mat p
  S$, such that $p = \alpha_p\circ \id_L$.
\end{oldresult}
%
\begin{proof}$\alpha_p$ is a graph morphism by construction of $\mat p E$. To
  show that $\alpha_p\ftype{\mat p S}S$ is an abstraction morphism we prove the
  properties of \dref{abstraction}.
%
\begin{enumerate}\noitemsep\notopsep
\item Let $v\in N_S$; then $\alpha^{-1}_p(v)=p^{-1}(v)\cup \set v$. Defining
  $i=\card{p^{-1}(v)}$ and using \pref{mult}.2 we obtain \
  \[ \textstyle\sum\mat p \nodeMult(\alpha^{-1}_p(v))
         = (\nodeMult_S(v)-i)+\meq i
         \subseteq \nodeMult_S(v) \enspace.
  \]
\item By construction of $\mat p \inMult$;
\item By construction of $\mat p E$.
\end{enumerate}
%
$\id_L$ is a concrete pre-shaping by construction of $\mat p S$, taking into
account that $p$ is already a shaping. Finally, $p=\alpha_p\circ \id_L$ is
immediate by the definition of $\alpha_p$.\qed
\end{proof}
%
\begin{oldresult}{Proposition}{p:mat}
  Let $L,G\in \DetGraph_\L$ and $S\in \Shape_\L$. For an arbitrary injective
  morphism $m\ftype L G$ and a shaping $s\ftype G S$, let $p= s\circ m$; then
  there is a shaping $t\ftype G{\mat p S}$ with $s= \alpha_p\circ t$ and
  $t\circ m= \id_L$.
\end{oldresult}
%
\begin{proof}
Note that $p$ is a pre-shaping by \pref{pre-shaping}, so the
materialisation $\mat p S$ is well-defined.  The required shaping $t$ is given
by
%
\[ t : v \mapsto \cases{m^{-1}(v) & if $v\in m(N_L)$ \cr s(v) &
otherwise.} 
\]
On the level of functions over node sets, we show $s= \alpha_p\circ t$ 
by a simple case distinction. Let $v\in N_G$ be arbitrary.
%
\begin{itemize}
\item If $v\in m(N_L)$ then $t(v)= m^{-1}(v)\in N_L$, implying
  $\alpha_p(t(v))= p(t(v))= s(m(m^{-1}(v)))= s(v)$.
\item If $v\notin m(N_L)$ then $t(v)= s(v)$, implying
  $\alpha_p(t(v))= \id_S(t(v))= s(v)$.
\end{itemize}
%
To see that $t\circ m= \id_L$ holds (as functions over node sets), let
$v\in N_L$ be arbitrary; then $m(v)\in m(N_L)$, hence $t(m(v))=
m^{-1}(m(v))= v$.
%
We now show that $t$ is a shaping of $G$ to $\mat p S$. 
\begin{itemize}
\item By construction, $t$ maps $N_G$ into $\mat p N$.

\item Let $(v,a,w)\in E_G$ be arbitrary. It follows that $(s(v), a,
  s(w)) \in E_S$ due to the fact that $s$ is a shaping; hence
  $(t(v), a, t(w))\in \alpha_p^{-1}(E_S)$ due to $s= \alpha_p
  \circ t$, proved above. To show that $(t(v), a, t(w))\in\mat p E$, we
  now only have to show that either $t(v) \notin N_L$ or $\nexists(t(v),
  a, w')\in E_L: w'\neq t(w)$. For this purpose assume $t(v)\in N_L$ and
  $(t(v), a, w')\in E_L$; then $(m(t(v)),a,m(w'))\in E_G$. It follows by
  construction of $t$ that $v\in m(N_L)$ and $v=m(t(v))$; hence $m(w')=w$
  due to the determinism of $G$, implying $t(m(w'))= t(w)$. Since
  $t(m(w'))= w'$ by construction of $t$, we are done.

\item Let $v\in \mat p N$ be arbitrary. We make the following case distinction:
\begin{itemize}
\item $v\in N_L$. By definition, $t^{-1}(v) = \set{m(v)}$;
  since $\mat p\nodeMult(v)=\meq1$, we are done.
\item $v\in N_S$. By definition, $t^{-1}(v) = s^{-1}(v)\setminus
  m(N_L)$. Since $m$ is injective, $\card{s^{-1}(v)\cap
  m(N_L)} = \card{m^{-1}(s^{-1}(v))} = \card{p^{-1}(v)}$. Since
  $s^{-1}(v):\nodeMult_S(v)$ by the fact that $s$ is a shaping, it
  follows by \pref{mult}.1 that $t^{-1}(v):\mat p\nodeMult(v)$.
\end{itemize}

\item Let $v\in N_G$ and $a\in \L$ be arbitrary, and define the 
  $a$-predecessors of $v$ in $G$ as $X=\genset{w\in
  N_G}{(w,a,v) \in E_G}$. By the fact that $s$ is a shaping it follows
  that $X: \inMult(s(v))(a)$. Due to $s=\alpha_p \circ t(v)$ it
  follows that $\mat p\inMult(t(v))= \mat p\inMult(\alpha_p(t(v)))=
  \mat p\inMult(s(v))$, hence $X: \mat p\inMult(t(v))(a)$.\qed
\end{itemize}
\end{proof}
%
\fref{transition} shows a diagram to clarify Theorems \ref{th:abs2con} and~\ref{th:con2abs}.
%
\begin{figure}[th]
  \[ \def\labelstyle{\textstyle}
     \xymatrix{L\ar[d]^m \ar@/_8mm/[dd]^*{c} \ar@/_16mm/[ddd]^p \\ 
               G \ar[r]^{P,m} \ar[d]^s & H \ar[d]^t \\
               \mat p S \ar[r]^{P,c} \ar[d]^{\alpha_p} & T \\
               S} 
  \]
  \smallcaptionsep
   \caption{Concrete and abstract transitions; visualization of Theorems \ref{th:abs2con} and~\ref{th:con2abs}.}
   \flabel{transition}
\end{figure}

\begin{oldresult}{Theorem}{th:abs2con}
  Let $P=(L,R)\in \Prod_\L$ and $S\in \Shape_\L$, and assume $S\trans{P,c} T$.
  For any shaping $s\ftype G S$, there exists a matching $m$ for $P$ in $G$ such
  that $c=s\circ m$, and $G\trans{P,m} H$ such that there is a shaping $t\ftype
  H T$.
\end{oldresult}
%
\begin{proof}
  Let $s\ftype G S$ be an arbitrary shaping.  By $S\trans{P,c}T$ we have that
  $c \ftype L {G_S}$ is a concrete pre-shaping.  Then, by \pref{concrete
    pre-shaping} there exists an injective morphism $m \ftype L G$ such that
  $c=s \circ m$. We show that $m$ is in fact a (concrete) matching.
%===========================
% NEW PART
%==========================
To show this, we  prove  that $m$ 
satisfies the conditions of \dref{graph-transformation}.
\begin{enumerate}
\item 
Let $(v,a,w) \in E_G$; this implies $(s(v),a,s(w)) \in E_S$. Since $c$
  is an abstract matching, we have:
\[\begin{array}{l}
s(v) \in c(N^\del) \vee s(w) \in c(N^\del) \implies (s(v),a,s(w)) \in
c(E^\del)
\\[\smallskipamount]
\quad\mbox{implying}
\\[\smallskipamount]
c^{-1}(s(v)) \in N^\del \vee c^{-1}(s(w)) \in N^\del \implies c^{-1}((s(v),a,s(w))) \in
E^\del 
\\[\smallskipamount]
\quad\mbox{implying [since $c^{-1}\circ s=m^{-1}$]}
\\[\smallskipamount]
m^{-1}(v) \in N^\del \vee m^{-1}(w) \in N^\del \implies m^{-1}((v,a,w)) \in E^\del 
\\[\smallskipamount]
\quad\mbox{implying}
\\[\smallskipamount]
v \in m(N^\del) \vee w \in m(N^\del) \implies (v,a,w) \in m(E^\del). 
\end{array}\] 
%
\item Again, if  $(v,a,w) \in E_G$, then $(s(v),a,s(w)) \in E_S$. Since $c$
  is an abstract matching, we have:
\[\begin{array}{l@{}}
s(v) \in c(N^\use) \wedge \exists (c^{-1}(s(v)), a,w') \in E^\new \implies
(s(v),a,s(w)) \in c(E^\del) 
\\[\smallskipamount]
\quad\mbox{implying}
\\[\smallskipamount]
c^{-1}(s(v)) \in N^\use \wedge \exists (c^{-1}(s(v)), a,w') \in E^\new \implies
c^{-1}((s(v),a,s(w))) \in E^\del
\\ [\smallskipamount]
\quad\mbox{implying [since $c^{-1}=m^{-1} \circ  s^{-1}$]} 
\\[\smallskipamount]
m^{-1}(v) \in N^\use \wedge \exists (m^{-1}(v), a,w') \in E^\new \implies
m^{-1}(v,a,w) \in E^\del  
\\[\smallskipamount]
\quad\mbox{implying}
\\[\smallskipamount]
v \in m(N^\use) \wedge \exists (m^{-1}(v), a,w') \in E^\new \implies
(v,a,w) \in m(E^\del)  
\end{array}\]
\end{enumerate}
This proves that $m$ is a (concrete) matching.  
%=================
% END NEW PART
%================
Hence, there exists a transition
$G\trans{P,m} H$ where by~\dref{graph-transformation} we have:
  \begin{eqnarray*}
    N_H & = & (N_G \setminus m(N^\del)) \cup N^{\new} \\
    E_H & = & (E_G \setminus m(E^\del)) \cup E^{\new}
  \end{eqnarray*}
where $N^\new$ and $E^\new$ are fresh by the assumption on the definition of
production rule.

It remains to prove that there exists a shaping $t\ftype H T$.
This is defined by the following node function:
%
\[   t:v  \mapsto  \left \{
     \begin{array}{ll}
        v & \mbox{if $v \in N^\new$}\\
       s(v) & \mbox{otherwise.}
      \end{array}\right.
\]
%
We prove that $t$ is indeed a shaping.
%
\begin{itemize}
\item $t$ is a graph morphism. To see this, let $(v,a,w) \in E_H$.  
If $(v,a,w) \in E_G \setminus m(E^\del)$ we have:
  $t(v,a,w)=s(v,a,w)=(s(v),a,s(w))=(t(v),a,t(w))$ since $s$ is a shape
  morphism.
Otherwise if $(v,a,w) \in E^\new$ then we have to distinguish several cases
depending whether $v,w$ belong to $N^\new$ or to $N^\use$.
In all cases, it is trivial to see that by construction we have
$t((v,a,w))=(f(v),a,f(w))=(t(v),a,t(w))$.
%
\item Now we show that conditions 1-3 of \dref{shaping} hold.
\begin{enumerate}
\item If $v \in N^\new$ then $|t^{-1}(v)|=1\in \nodeMult_{T}(v)$.  If $v \in
  N_{S}$ then, since $s$ is a shaping, we have $|t^{-1}(v)|=|s^{-1}(v)|\in
  \nodeMult_{S}(v)=\nodeMult_{T}(v)$.
  
\item Let $v \in N_H$ and $a \in \L$. If $v \in N^\new$ then
  $\inMult_{T}(t(v))(a)= \inMult_{T}(v)(a)$, which equals
  $\meq{\card{\genset{w}{(w,a,v) \in E^\new}}}$.  However, since $v \in N^\new$
  we have $\genset{w}{(w,a,v) \in E^\new}=\genset{w}{(w,a,v) \in E_H}$.
  $\subseteq$ is trivial. We show $\supseteq$ by contradiction.  Assume
  $\exists (w,a,v) \in E_H \setminus E^\new$ then $(w,a,v) \in E_G \setminus
  m(E^\del)$ and therefore $v \notin N^\new$ which is indeed a contradiction.
  Hence, we conclude that $\genset{w}{(w,a,v) \in E_H}:\inMult_{T}(t(v))(a)$.

  If $v \in N_G \setminus m(N^\del)$ then 
  \begin{eqnarray*}
  \inMult_{T}(t(v))(a) 
  & = & \inMult_{S}(s(v))(a)
         - \card{\genset{w}{(w,a,s(v)) \in c(E^\del)}} \\
  &   & {}+ \meq{\card{\genset{w}{(w,a,s(v)) \in E^\new}}} \enspace.
  \end{eqnarray*}
%
  Due to the fact that $c$ and $m$ are injective morphisms, and moreover
  $m^{-1}=c^{-1}\circ s$, we have
%
  \begin{eqnarray*}
  \lefteqn{\card{\genset{w \in N_H}{(w,a,v) \in E_H}}} \\
  & = & \card{\genset{w \in N_G}{(w,a,v) \in E_G}} 
         - \card{\genset{w \in N_G}{(w,a,v) \in m(E^\del)}} \\
  &   & {}+ \card{\genset{w \in N_G}{(w,a,v) \in E^\new}} \\
  & = & \card{\genset{w \in N_G}{(w,a,v) \in E_G}}
         - \card{\genset{w \in N_{T}}{(w,a,s(v)) \in c(E^\del)}} \\
  &   & {}+ \card{\genset{w \in N_{T}}{(w,a,s(v)) \in E^\new}} \enspace.
  \end{eqnarray*}
%
  Since $s$ is a shaping, we have $\card{\genset{w \in N_G}{(w,a,v) \in
  E_G}}\in \inMult_{S}(s(v))(a)$; hence we may conclude $\card{\genset{w \in
  N_H}{(w,a,v) \in E_H}}\in \inMult_{T}(t(v))(a)$.
%
\item Let $v \in N_H, \ a \in \L$ and $(t(v),a,w) \in E_{T}$.  If
  $(t(v),a,w) \in E^\new$ then by construction $(t(v),a,w) \in E_H$.
  
  If $(t(v),a,w) \in E_{S} \setminus c(E^\del)$ then $t(v) \in N_{S}
  \setminus c(N^\del)$.  By definition of $t$ it follows that $v \notin
  N^\new$ which implies $t(v)=s(v)$.  Therefore we have
  $(t(v),a,w)=(s(v),a,w)$.  Since $s$ is a shaping by hypothesis, then there
  exists $(v,a,w') \in E_G$ for some $w'$ such that $s(w')=w$.  Thus to show
  that $(v,a,w') \in E_H$ it remains to be proved that $(v,a,w') \notin
  m(E^\del)$.  We prove that by contradiction. Assume $(v,a,w') \in m(E^\del)$
  then $m^{-1}((v,a,w')) \in E^\del$.  Since $m^{-1}=c^{-1}\circ s$ then it
  follows $s(v,a,w') \in c(E^\del)$ which implies $(s(v),a,s(w')) \in
  c(E^\del)$. This finally implies $(t(v),a,w) \in c(E^\del)$ that contradicts
  our initial assumption.
 \end{enumerate}
%
 Hence, we conclude that $t$ is a shaping.\qed
\end{itemize}
\end{proof}

\begin{oldresult}{Theorem}{th:con2abs}
  Let $P=(L,R)\in \Prod_\L$ and $G\in \DetGraph_\L$, and assume $G\trans{P,m} H$.
  For any shaping $s\ftype G S$ such that $c=s\circ m$ is concrete,
  $S\trans{P,c} T$ such that there is a shaping $t\ftype H T$.
\end{oldresult}
%
\begin{proof}
  Since there exists $m \ftype L G$ and $s \ftype G S$ then by
  \pref{pre-shaping} $s \circ m \ftype L S$ is a pre-shaping.  Let $c=s \circ
  m$ be concrete. Because $m$ is a concrete matching, we can prove that $c$ is
  an abstract matching for $P$ in the graph part of $S$.  
%===========================
%   NEW PART
%===========================
  To show this we prove condition 1 and 2 of \dref{graph-transformation}. Let
  $(v,a,w) \in E_S$, we have $(c^{-1}(v),a,c^{-1}(w)) \in E_L$. Since
  $c^{-1}=m^{-1} \circ s^{-1}$ we have:
\[
(s^{-1}(v),a,s^{-1}(w)) \in E_G.
\] 
Since $m$ is a concrete matching we have: 
\begin{enumerate}
\item For the first condition
\[
\begin{array}{l}
 s^{-1}(v) \in m(E^\del) \vee s^{-1}(w) \in m(E^\del)
                \implies (s^{-1}(v),a,s^{-1}(w)) \in m(E^\del) 
\\[\smallskipamount]
\quad\mbox{implying}
\\[\smallskipamount]
 m^{-1}(s^{-1}(v)) \in E^\del \vee m^{-1}(s^{-1}(w)) \in E^\del \\
\qquad\qquad {} \implies (m^{-1}(s^{-1}(v)),a,m^{-1}(s^{-1}(w))) \in E^\del
\\[\smallskipamount]
\quad\mbox{implying [since $c^{-1}=m^{-1}\circ s^{-1}$]}
\\[\smallskipamount]
 c^{-1}(v) \in E^\del \vee c^{-1}(w) \in E^\del
                \implies (c^{-1}(v),a,c^{-1}(w)) \in E^\del
\\[\smallskipamount]
\quad\mbox{implying}
\\[\smallskipamount]
 v \in c(E^\del) \vee w \in c(E^\del)
                \implies (v,a,w) \in c(E^\del) 
\end{array}
\]
%
\item For the second condition:
\[
\begin{array}{ll}
s^{-1}(v) \in m(N^\use) \wedge \exists (m^{-1}(s^{-1}(s)),a,w') \in E^\new \\
\qquad\qquad {} \implies (s^{-1}(v),a,s^{-1}(w)) \in m(E^\del)
\\[\smallskipamount]
\quad\mbox{implying}
\\[\smallskipamount]
v \in s \circ m(N^\use) \wedge \exists (c^{-1}(s),a,w') \in E^\new
                \implies (v,a,w) \in s \circ m(E^\del)
\\[\smallskipamount]
\quad\mbox{implying}
\\[\smallskipamount]
v \in c(N^\use) \wedge \exists (c^{-1}(s),a,w') \in E^\new
                \implies (v,a,w) \in c(E^\del)
\end{array}
\]
\end{enumerate}
Therefore we conclude that $c$ is an abstract matching.
%=================================
% END NEW PART
%==================================
  Then, by \dref{shape-transformation} there
  exists a transition $S \trans{P,c} T$.  Moreover, The target graphs $H$ and
  $T$ of the concrete and the abstract transition are defined according to
  \dref{graph-transformation} and \dref{shape-transformation}.  Let $t\ftype H
  T$ be defined as in the proof of Theorem~\ref{th:abs2con}; as shown in that
  proof, $t$ is a shaping.\qed
\end{proof}

\begin{oldresult}{Theorem}{th:transitions}
Let $\Pi$ be a set of production rules and $I\in \DetGraph$; let
$\GTS(\Pi,I)=(\bG,{\trans{}})$ and $\STS(\Pi, \GCanon(I))=(\bS,{\trans{}})$.
\begin{enumerate}\noitemsep\smalltopsep
\item $\GCanon(\bG) \subseteq \bS$ and $\bS$ is finite;
\item For all $G,H\in \bG$, $G\trans P H$ implies $\GCanon(G) \trans P
  \GCanon(H)$.
\item For all $S,T\in \bS$ such that $S\trans{P} T$, there are $G',H'\in
  \DetGraph$ with a shaping $s\ftype{G'}S$, such that $G'\trans{P} H'$.
\end{enumerate}
\end{oldresult}
%
\begin{proof} Let $S=\GCanon(G)$.
\begin{enumerate}
\item $\GCanon(\bG)\subseteq \bS$ follows from the next item. The finiteness of 
  $\bS$ is a consequence of the finiteness of $\Canon_\L$, proved in
  \cite{Rens04-esop}.
\item Let $P=(L,R)$ and assume $m$ is the matching for $G\trans P H$. By
  construction of $S$, there is a shaping $s\ftype G S$ (see
  \thref{S^canon_G}). Due to \pref{pre-shaping}, $p=s\circ m$ is a pre-shaping
  of $L$ in $S$; hence due to \pref{mat construction}, there is a concrete
  pre-shaping $\id_L\ftype L{\mat p S}$. Hence due to \thref{con2abs}, there is
  an abstract shape transformation $\mat p S\trans{P,\id_L} T$ with a shaping
  $t\ftype H T$. By \thref{canon(S)} we have $\GCanon(H)\in \SCanon(T)$. It
  follows that, by definition, $S\trans P T$.
\item Let $P=(L,R)$. By definition of shape transitions, $\mat p S$ is
  consistent, $\mat p S\trans{P,\id_L} S'$ and $T\in \GCanon(S')$ for some
  pre-shaping $p\ftype L S$. It follows that there is a shaping $s'\ftype G{\mat
  p S}$; by \pref{abstraction}, $s=\alpha_p\circ s'$ is then a shaping of $G$ in
  $S$. By \thref{abs2con}, there exists a matching $m$ for $P$ in $G$ such that
  $\id_L=s\circ m$ and $G\trans{P,m} H$.  \qed
\end{enumerate}
\end{proof}

%%% Local Variables: 
%%% mode: latex
%%% TeX-master: "agt"
%%% End: 
