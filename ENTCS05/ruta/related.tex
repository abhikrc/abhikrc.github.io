Tin this Section we briefly describe the more relevant proposals in the field of web application verification.
Some approaches consider the web similar to a database, hence propose conceptual models of its structure; more recent approaches focus on web applications under a web engineering point of view. A complete review of all the modeling techniques is in \cite{fraternali-99}.

HDM \cite{gar-pao-93} is one of the first model-driven design of hypermedia applications; successive proposals are RMM\cite{isa-sto-95}, Strudel \cite{fer-98} and Araneus\cite{atz-mec-mer-98}. They all build on the HDM model and support specific navigation constructs. In particular, Araneus describes the data structure based on the entity relationship model.  

In the second perspective, that is considering a Web Application as a software object, several modeling techniques have been adopted. Conallen \cite{cona-02} proposes a UML-based methodology. The main benefit of the method is the feature which allows to represent all the components of a Web Application using a standard UML notation. OOHDM \cite{ros-sch-02} is an object-oriented method to represent design structure in WAs. The method considers Web Applications as navigational views over an object model and provides some basic constructs for designing the navigation model. UWE \cite{koch-kraus-02} is an object-oriented, iterative and incremental approach based on the UML. 
 
WebML \cite{cer-fra-mat-02} introduces graphical and XML representation of concepts for designing Web Applications. Anyway, all the proposals are modeling techniques. To perform verification of a WA it is necessary to use verification or testing techniques. 

The method proposed in \cite{ric-ton-02} is based on a UML model of WAs and considers the testing and validation of the developed web system. In \cite{antoniol}, a Web Application analysis based on queue models is proposed. Finally, in \cite{dilucca} the authors verify the correct use of duplicated pages inside a web constructed using HTML language and ASP code. Once again the proposed method does not consider a formal approach.

On the other hand, model checking based on a $\mu-calculus$ language has been used in \cite{de-alfaro-01}, but the approach does not present the analysis of dynamic pages. Anyway in this work the author considers a model of the web like a graph in which states are pages and transitions between states are hyperlinks in the pages. Hence hyperlinks cannot be qualified by properties as we do.

In \cite{stotts-98} the automata are used to outline the framework of the links in a hypertext. Hence a branching temporal logic (HyperText Logic) HTL is defined. By means of it a sequence of transitions between states in the automata can be described. The logic is also used to verify the propositions of the temporal logic, but again dynamic pages are not considered.

In \cite{veriweb} a tool for automatically discovering and systematically exploring web-site execution paths is proposed: it is a spider-like program which follows all the possible static links in every HTML web page and surfs through all dynamic components of a web application, including both submission/execution forms and client-side scripts. During web site examination, the tool allows to check for many kind of errors, in single web pages as well as in a navigation paths, by means of a regression test. However, the WA must be already implemented in HTML to perform all tests. Hence possible discovered errors have a great cost of repairing.