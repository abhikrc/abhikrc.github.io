Nowadays, Web Applications (WAs) are a powerful support to human activities, especially in business, scientific or medical fields. The relevance that WAs are assuming asks for an increase of their reliability, flexibility and security; there is the need of methods and tools for supporting the WA design and performing both an automated verification and validation of it.
In this context, Symbolic Model Checking techniques \cite{katoen} can be a valid choice within verification methods; notice that with respect to other test and simulation approaches, Model Checking ones do not need a preliminary interaction with the user for selecting inputs as test case, hence it could be used in design phase allowing to locate mistakes or malfunctions. This strategy should made a more accurate design, before implementing WA as prototype to be tested; then it saves time and reduces costs in development cycle.

According to above considerations, first of all, it is necessary to transform a generic Web Application design in a  model to be verified by means of NuSMV. In our previous papers \cite{seke-02}\cite{csmr-03}\cite{disc-doni-mong-tota-cast} we built a mathematical model, where a WA design is represented by a FSM, where windows, links, pages and actions are states. For the sake of simplicity we use a limited number of states because we consider only the logical-functional structure of web pages ignoring the interactive aspects to avoid a possible uncontrolled increase in state number. Secondly, after automatically verifying specifications about correctness, the obtained SMV model is subsequently refined: explanations provided by model checker are analyzed to establish if there is a violation in Computational Tree Logic (CTL) \cite{katoen} axioms. Finally errors in WA design are located and corrected.

In the proposed approach, the SMV model is derived from a WA navigation model which is drawn in Unified Modeling Language (UML), according to extensions proposed by Conallen \cite{cona-02} (they result particularly useful in modeling web systems, characterized by client/server interactions as well as frame-based navigation). UML diagrams provide a valid support to verify WA requirements, however they need to be turned into our corresponding mathematical model. 

Hence, in this paper we introduce a tool, able to transform UML diagrams in XMI files and to turn them into corresponding Web Application Graphs (WAGs). Then the tool translates the WAG in a SMV model finally used as input for the model checker NuSMV. It automatically performs verification of CTL specifications. 

Verification of properties of a Web Application by means of Symbolic Model Checking techniques is very useful in web planning, because it determines a gradual refinement of the WA design, before it was definitely implemented.