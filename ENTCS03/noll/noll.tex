\documentclass{entcs}

\usepackage{entcsmacro}
%\usepackage{graphicx}

\usepackage{listings}
\usepackage{xspace}
\usepackage{semantic}
\usepackage{pstricks}
\usepackage{pst-tree}
%\usepackage{amsmath}
%\usepackage{i2ams}
%\usepackage{epsfig}
%\usepackage{url}

% some macros ======

\def\exsty#1{\mbox{\texttt{#1}}} %defines the style for examples and
\def\per{\mbox{\textbf{.}}}
\def\qed{\hfill\fbox{\space}\\ \vskip 1\baselineskip}

\newcommand{\dd}[3]{#1_{1}{#3}\dots{#3}#1_{#2}} % makes x1 o ... o xn
%\newtheorem{definition}{Definition}
%\newtheorem{theorem}{Theorem}
%\newtheorem{proposition}[theorem]{Proposition}

\def\ie{\textit{i.e.},\ }
\def\eg{\textit{e.g.},\ }
\def\ctl{CTL\xspace}
\def\smv{SMV\xspace}
\def\wrt{with reference to\ }
\def\centr#1{centr(#1)} % centroid of #1
\def\p{\textbf{\textrm{p}}}    % style for points (centroids)
\def\u{\textbf{\textrm{u}}}
\def\v{\textbf{\textrm{v}}}
\newcommand{\llist}[1]{\langle #1 \rangle}
\newcommand{\AND}{\ensuremath{\sqcap}}
\newcommand{\ALL}[2]{\exsty{ALL} #1 \per #2}
\newcommand{\SOME}[2]{\exists (#1) \per #2}
%\newcommand{\proof}[1]{\textsl{Proof.} #1 \qed}

% macro per le bmp
\def\setbmp#1#2#3#4{\vskip#3\relax\noindent\hskip#1\relax
 \special{bmp:#4 x=#2, y=#3}}
\def\centerbmp#1#2#3{\vskip#2\relax\centerline{\hbox to#1{\special
  {bmp:#3 x=#1, y=#2}\hfill}}}

\newcommand{\fourcomponent}[4]{\langle #1,#2,#3,#4 \rangle}
\newcommand{\twocomponent}[2]{\langle #1,#2 \rangle}
\def\I{{\ensuremath{\cal I}}}
\newcommand{\Int}[1]{#1^\I}             % argument^I
                                        % interpretation function
\newcommand{\INT}[1]{(#1)^\I}           % (argument)^I,
                                        % interpretation function


\def\lastname{Noll}



\begin{document} 
\begin{frontmatter}
  \title{Equational Abstractions for Model Checking \Erlang Programs}
  \author{Thomas Noll\thanksref{myemail}}
  \address{Lehrstuhl f\"ur Informatik II\\ Aachen University\\
    52056 Aachen, Germany}
  \thanks[myemail]{Email:
    \href{mailto:noll@cs.rwth-aachen.de} {\texttt{\normalshape
        noll@cs.rwth-aachen.de}}}
\begin{abstract}
  This paper provides a contribution to the formal verification of
  programs written in the concurrent functional programming language
  \Erlang, which is designed for telecommunication applications. It
  presents a formal description of this language in Rewriting Logic,
  a unified semantic framework for concurrency which is semantically
  founded on conditional term rewriting modulo equational theories.
  In particular it demonstrates the use of equations for defining
  abstraction mappings which reduce the state space of the system.
\end{abstract}
\begin{keyword}
  Software Verification, Concurrent Functional Programming,
  Rewriting Logic
\end{keyword}
\end{frontmatter}



\section{Introduction}
\label{SctIntro}

In this paper we address the software verification issue in the context of the
functional
programming language \Erlang \cite{AVWW96}, which was developed by the
Ericsson corporation to address the complexities of developing
large--scale programs within a concurrent and distributed setting.
Our interest in this language is twofold. On the one hand, it is often and
successfully used in the design and implementation of
telecommunication systems. On the other hand, its relatively compact syntax and its clean semantics supports the application of formal reasoning methods.

Due to the presence of unbounded data
structures and of dynamic process spawning, \Erlang programs usually
induce infinite--state systems. It is therefore natural to employ
interactive \emph{theorem--proving assistants} such as the \EVT \Erlang
Verification Tool \cite{FGN01,FGN02} to establish the desired system
properties.

Here we follow an alternative approach in which we try to employ
fully--automatic \emph{model--checking techniques} to establish correctness
properties of communication systems implemented in \Erlang. Here we concentrate
on the first part of the verification procedure, the construction of the
(transition--system) model to be checked.

More concretely, we formally describe \Erlang using the \emph{Rewriting
Logic} framework, which was proposed in \cite{Mes92} as a unified
semantic framework for concurrency.  It has proven to be an adequate
modeling formalism for many concrete specification and programming
languages \cite{MM02}.  In this approach the state of a system is
represented by an equivalence class of terms modulo a given set of
equations, and transitions correspond to rewriting operations on the
representatives. Hence Rewriting Logic supports both the definition of
programming formalisms and, by employing (equational) term rewriting
methods, the execution or simulation of concrete systems. We will see that the
equations can be used to define abstraction mappings which reduce the state
space of the system. In particular we will discuss examples where it is
possible to shrink a system with infinitely many states to a finite one.

By employing an executable implementation of the Rewriting Logic framework such 
as the ELAN tool \cite{ELAN} it is possible to automatically derive the 
transition system of a given \Erlang program. Thereafter model--checking tools 
such as \Truth \cite{LLNT99} can be used to automatically verify that the 
system meets certain conditions given as formulae of some mathematical logic.
The latter, however, is outside the scope of this article.

The remainder of this paper is organized as follows.
Section~\ref{SctErlang} introduces the \Erlang programming language by
sketching its syntactic constructs and their intuitive
meaning. Section~\ref{SctRL} introduces the Rewriting Logic framework, and 
employs it to define the transition--system semantics of \Erlang. Then 
Section~\ref{SctEqu} demonstrates the use of equations to reduce the size of 
the transition system, and finally Section~\ref{SctConcl} concludes with some 
remarks.



\section{The Erlang Programming Language}
\label{SctErlang}

\Erlang/\OTP is a programming platform providing the necessary
functionality for programming open distributed (telecommunication)
systems: the language \Erlang with support for concurrency, and the
\OTP (Open Telecom Platform) middleware providing ready--to--use
components (libraries) and services such as e.g.\ a distributed data
base manager, support for ``hot code replacement'', and design
guidelines for using the components.

In the following we consider a core fragment of the \Erlang programming
language which supports the implementation of dynamic networks of processes
operating on data types such as atomic constants (atoms), integers,
lists, tuples, and process identifiers (pids), using asynchronous,
call--by--value communication via unbounded ordered message queues called
mailboxes.  Real \Erlang has several additional features such as modules,
distribution of processes (onto nodes), and support for robust programming 
and for interoperation with non--\Erlang code written in, e.g., C or Java.

Besides \Erlang \emph{expressions} $e$ we operate with the syntactical
categories of \emph{matching clauses} $\mathit{cs}$, \emph{patterns} $p$,
and \emph{values} $v$.  The abstract syntax of \Erlang expressions
is summarized as follows:
\[\begin{array}{r@{~}r@{~}l}
%    e & ::=  & \BEGINEND{e_1, \ldots, e_n} \\
%      & \mid & e_1, e_2 \\
%      & \mid & e(e_1, \ldots, e_n) \\
%      & \mid & \CASEOF{e}{\mathit{cs}} \\
%      & \mid & p = e \\
%      & \mid & e_1 \BANG e_2 \\
%      & \mid & \RECEIVEEND{\mathit{cs}} \\
%      & \mid & \mathit{op}(e_1,\ldots,e_n) \\
%      & \mid & \SPAWN(e_1, e_2) \\
%      & \mid & \SELF() \\
%      & \mid & V \\
    e & ::=  & e_1, e_2 \mid 
               e(e_1, \ldots, e_n) \mid
               \CASEOF{e}{\mathit{cs}} \mid p = e \mid e_1 \BANG e_2 \\[-1ex]
      & \mid & \RECEIVEEND{\mathit{cs}} \mid \mathit{op}(e_1,\ldots,e_n) \mid 
               \SPAWN(e_1, e_2)
               \SELF() \mid X \\
    \mathit{cs} & ::= & p_1 \ARROW e_1; \ldots; p_n \ARROW e_n \\
    p & ::=  & \mathit{op}(p_1,\ldots,p_n) \mid X \\
    v & ::=  & \mathit{op}(v_1,\ldots,v_n)
  \end{array}
\]
Here $X$ ranges over \Erlang variables, and $\mathit{op}$ ranges over a
set of primitive constants and operations including tupling $\{e_1,e_2\}$,
list prefix $[e_1|e_2]$, the empty list $[\,]$, integers, pid constants,
and atoms.

The functional sublanguage of \Erlang is rather standard: atoms,
integers, lists and tuples are value constructors; $e_1, e_2$ denotes 
sequential composition; and
$e(e_1,\ldots,e_n)$ represents a function call. An expression of the form
$\CASEOF{e}{p_1 \ARROW e_1; \ldots; p_n \ARROW e_n}$ involves
matching:  the value that $e$ evaluates to is matched sequentially
against the patterns $p_i$. If this succeeds, evaluation continues with
$e_i$ where the variables bound by $p_i$ are correspondingly
instantiated. The same is true for the assignment $p = e$ where a
runtime error is raised if the value of $e$ does not match $p$, and
where this value is returned as the result otherwise.

The constructs involving non--functional behavior (i.e., side effects)
are $e_1 \BANG e_2$ which denotes an output step, sending the value of
$e_2$ asynchronously to the process identified by $e_1$, whereas
$\RECEIVEEND{\mathit{cs}}$ inspects the mailbox $q$ of the local
process and retrieves (and removes) the first element in $q$ that
matches any pattern in $\mathit{cs}$. Once such an element $v$ has been
found, evaluation proceeds analogously to $\CASEOF{v}{\mathit{cs}}$.
$\SPAWN(e_1, e_2)$ dynamically creates a new process in which the
function given by $e_1$ is applied to the arguments given by the list
$e_2$, and $\SELF()$ returns the pid of the local process.

As an introductory example we consider a short \Erlang program which
implements a simple resource locker, i.e., an arbiter which, upon receiving
corresponding requests from client processes (two in this case), grants access
to a single resource. An extended version of the algorithm is presented in
\cite{AED02}, which in addition is capable of handling several resources from a
given, finite set.

An \Erlang program consists of a set of modules. Each module basically
contains a list of function declarations. In our example the system is defined
in one module. It is initialized using the \code{start} function, which,
according to the \code{export} declaration, is the only function accessible
from outside the \code{locker} module. By calling the corresponding startup
functions, it generates three new processes: one locker and two clients. The
actual process creation is performed by the \SPAWN builtin function which
receives the module identifier and the name of the function to be invoked in
the new process, together with its arguments.

The locker process runs the \code{locker} function in a non--terminating loop.
It employs the \RECEIVE construct to check whether a request message has
arrived. The latter is expected to be a pair composed of a \code{request} tag
and a client process identifier (which is matched by the variable
\code{Client}).  The client is then granted access to the resource by sending
an \code{ok} flag. Finally, after receiving the release message from the
respective client, the locker returns to its initial state.

A client process exhibits the complementary behavior. By issuing a request,
it demands access to the resource. Here, the \SELF builtin function returns
the process identifier (pid) of the client process, which is then used by the
locker process as a handle to the client. After receiving the \code{ok}
message it accesses the resource, and releases it afterwards.

\begin{minipage}[t]{0.45\textwidth}
\begin{lstlisting}[gobble=2]{}
  -module(locker).
  -export([start/0]).

  start() ->
    Locker = start_locker(),
    start_client(Locker),
    start_client(Locker).

  start_locker() ->
    spawn(locker, []).

  locker() ->
    receive
      {request, Client} ->
        Client!ok,
        receive
          {release, Client} ->
            locker()
        end
    end.
\end{lstlisting}
\end{minipage}
\hfill
\begin{minipage}[t]{0.45\textwidth}
\begin{lstlisting}[gobble=2]{}
  start_client(Locker) ->
    spawn(client, [Locker]).

  client(Locker) ->
    Locker!{request, self()},
    receive
      ok ->
        % critical section
        Locker!{release,self()},
        client(Locker)
    end.
\end{lstlisting}
\end{minipage}

\medskip
The desirable correctness properties of such a system are straightforward:
\begin{description}
  \item[no deadlock:] there exists no cyclic chain of processes waiting for
    each other to continue, i.e., the locker should always be enabled to 
    receive a new request or a release, 
  \item[mutual exclusion:] no two clients should gain access to
    the resource at the same time, and 
  \item[no starvation:] all clients enabled to enter the critical section
    should eventually be granted their demanded access.
\end{description}

Later we will see how to check these properties by constructing the transition
system of the above program. Using the latter, the absence of deadlocks can be
verified by showing that the system can always proceed, i.e., that every state
has a direct successor. Mutual exclusion can be established by proving that
between the receptions of two successive \code{request} messages by the locker
there must always occur a \code{release} operation.

Guaranteeing the no--starvation property, however, needs additional
assumptions about the behavior of the process scheduler. In principle it
could happen that one of the client processes indefinitely remains in its
initial state, i.e., is never scheduled for sending the request message to the
locker. This situation, however, is excluded by the requirements which have to
be met by all \Erlang implementations (cf.\ \cite[Sct.~5.6]{AVWW96}). First,
the scheduling algorithm must be \emph{fair}, i.e., any process which is
enabled for execution will eventually be run. Moreover no process will be
allowed to block the machine for a longer period. This is postulated since
\Erlang should be suitable for \emph{soft real--time} applications where
response times must lie in the order of milliseconds.

As soon as the client has sent the request to the locker, eventual access to
the resource is guaranteed since the implementation of the (locker) mailbox in
the \Erlang runtime system follows a FIFO policy.



\section{Formal Semantics of \Erlang}
\label{SctRL}

The starting point of any kind of rigorous verification is a formal
semantics. Here we use an operational semantics by associating a
transition system with an \Erlang program, giving a precise account of
its possible computations. 

\subsection{The Rewriting Logic Framework}

The Rewriting Logic framework has been presented by J.~Meseguer in
\cite{Mes92}. An introduction to this approach together with an extensive
bibliography can be found in \cite{MM02}. 

Rewriting Logic is intended to serve as a unifying mathematical model
and uses notions from rewrite systems over equational theories. It aims at a
separate description of the static and of the dynamic aspects of a concurrent
system. More exactly, it distinguishes the laws describing the structure of
the states of the system from the rules which specify its possible
transitions. The two aspects are respectively formalized as a set of equations
and as a (conditional) term rewriting system. Both structures operate on
states, represented as (equivalence classes of) $\Sig$--terms where $\Sig$ is
the signature of the specification language under consideration. Since a single
transition may comprise
several independent rewriting steps, concurrent behavior can explicitly be
modelled in this way.

More concretely, in Meseguer's approach the syntax of Rewriting Logic is given
by a \emph{rewrite theory} $\ORT = (\Sig, \Equ, L, \Trans)$ where
\begin{itemize}
  \item $\Sig$ is a signature, i.e., a ranked alphabet of function symbols,
  \item $\Equ \subseteq \Term{\Sig}{\Var} \times \Term{\Sig}{\Var}$ is a
    finite set of equations over the set $\Term{\Sig}{\Var}$ of $\Sig$--terms
    with variables from a given set $\Var$,
  \item $L$ is a finite set of symbols called \emph{(rule) labels}, and
  \item $\Trans \subseteq L \times (\Term{\Sig}{\Var} \times
    \Term{\Sig}{\Var})^+$ is a  finite set of \emph{(conditional)
    transition rules} where each
    $(\rho, (l \trans{} r) (c_1 \trans{} d_1) \ldots (c_k \trans{} d_k))
    \in \Trans$ is represented as
    \[\inference[\rl{$\rho$}]{
        c_1 \trans{} d_1 & \ldots & c_k \trans{} d_k
      }{
        l \trans{} r
      }
    \]
    Here $l$ and $r$ are called the \emph{left--hand side} and the
    \emph{right--hand side}, respectively, of the rule.
    The upper part is called its \emph{condition}, and
    may sometimes be abbreviated with the letter $C$. If $k = 0$,
    then the rule is called \emph{unconditional}.
\end{itemize}

With regard to the semantics of Rewriting Logic, Meseguer defines that a
rewrite theory $\ORT$ \emph{entails} a \emph{sequent} $\classE{s} \trans{}
\classE{t}$ and writes 
\[\ORT \vdash \classE{s} \trans{} \classE{t}\]
if this sequent can be obtained by a finite number of applications of certain
\emph{rules of deduction} which specify how to apply the above transition rules.
In this way it is possible to reason about
concurrent systems whose states are presented by terms and which are evolving
by means of transitions. Here, the states are structured according to the
signature, and the transition rules specify the local
transitions in this structure whereas the deduction rules allow to reason
about the overall behavior of the concurrent system given the local
transformations.

Equations are used to identify terms which differ only in their syntactic 
representation. Later we will see that they can also be employed to define 
abstraction mappings on the state space.

It is a fact, however, that (conditional) term rewriting modulo equational 
theories is generally too complex or even undecidable. Hence it is not
possible to admit arbitrary equations in $\Equ$. Following the ideas of 
P.~Viry in \cite{Vir94}, we therefore propose to decompose $\Equ$ into a 
set of directed equations 
(that is, a term rewriting system), $\ER$, and into a set $\AC$ expressing
associativity and commutativity of certain binary operators in $\Sig$. Given 
that $\ER$ is terminating modulo $\AC$, 
rewriting by $\Trans$ modulo $\Equ$ can be implemented by a combination of
normalizing by $\ER$ and rewriting by $\Trans$, both modulo $\AC$. Here
the steps induced by $\Trans$ represent the actual state transitions of
the system while the reductions defined by $\ER$ have to be considered as
internal, non--observable computations.


\subsection{A Rewriting Logic Specification of \Erlang}

As mentioned earlier, the \Erlang runtime system 
maintains a set of user \emph{processes}.
Any such process consists of three components:
an \Erlang \emph{expression} which has to be evaluated,
a \emph{process identifier} (pid), which uniquely identifies the
respective process, and which is internally determined by the system, and
a \emph{mailbox} for incoming messages, which is essentially a list of
\Erlang values.

Moreover we will attribute a transition label
and a current evaluation environment to a process. The former is used to
indicate the type of the transition which lead to the current state. The
latter stores the bindings between the \Erlang variables and
the values assigned to them. It is modified by
an assignment or by a pattern matching
operation. Syntactic restrictions imposed on the code guarantee that every
occurrence of a variable name lies within the scope of a binding
operation.

As mentioned earlier, there can be several processes running concurrently
in a system. We therefore introduce the notion of a \emph{process system},
which is just a set of concurrent processes, and which constitutes a state
in the transition--system semantics:
\[\begin{array}[t]{@{}r@{~}c@{~}l@{}}
    S & = &
    \{\process{\alpha}{e}{i}{q}{\rho} \mid
    \begin{array}[t]{@{}l@{}}
      \alpha \mbox{ label}, e \mbox{ expression}, i \mbox{ pid}, \\[-1.5ex]
      q \mbox{ mailbox}, \rho \mbox{ environment}\}
    \end{array} \\[-1ex]
    & \cup &
    \{\dead{i} \mid i \mbox{ pid}\} \\[-1ex]
    & \cup &
    \{s_1 \parallel s_2 \mid s_1, s_2 \in S\}
  \end{array}
\]

The construct $\dead{i}$ denotes a \emph{dead} process whose actual computation
has been terminated. In contrast a process of the first form is called 
\emph{live}.
Both live and dead processes be combined using the associative and commutative
\emph{parallel composition operator} $\parallel$ to obtain a concurrent
process system. This, however, makes only sense if every process is uniquely
identified by its pid. We therefore call a process system \emph{well formed}
if all pids which occur in the process tuples are distinct, and assume every
process system to be well formed from now on.

The states of both single processes and process systems evolve over time (e.g.,
the expression of a process changes due to evaluation, or the mailbox stores an
incoming message). The transition labels attached to the processes reflect
the kind of the transition that lead to the current state. 
These include the label $\tau$ which expresses that a step not
involving a side effect was taken, such as the evaluation of a builtin
function like \code{+}. Other transition labels such as
$\MSG{j}{\code{42}}$ describe the sending of the value \code{42} to the process
identified by $j$, while $\SPN{\code{foo}}{\code{[1,2]}}{j}$
represents the call of a function involving side effects (the \SPAWN function
in this case). Here it is important to observe that transition labels are not
just comment--like annotations to the processes. Rather they implement a means
of communication between the two levels of the semantics, single processes and
concurrent systems, thus determining the transitions a concurrent process
system can take as a whole.

To simplify the presentation we refrain from formalizing those aspects of
\Erlang which are related to its module system. Thus we always assume that
the body of a function can somehow be determined from its module identifier and
its name.

\subsection{The Equational Theory}

The next step involves the definition of the set of equations, $\Equ$, of our
rewrite theory. In the $\AC$ part we only need to declare the parallel
operator $\parallel$ to be associative and commutative. The set of directed
equations, $\ER$, is used to model some auxiliary functions which are employed 
in the transition rules. Due to lack of space we
refrain from showing the specification and refer to \cite{Ami02}
instead. Let us just mention that we obtain a term rewriting system which is
convergent modulo $\AC$.

\subsection{The Transition Rules}

The most important part of our definition is the formalization of the
operational behavior of \Erlang process systems by conditional transition
rules. To obtain a cleaner structure we decompose $\Trans$ into two disjoint
subsets:
\[\Trans = \TPrc \cup \TSys.\]
Here, $\TPrc$ contains the so--called \emph{process--level rules} which
operate on single processes while $\TSys$, the set of \emph{system--level
rules}, deals with concurrent process systems. In the following we present some 
examples from both categories, again referring to \cite{Ami02} for the 
complete definition.

As before we will use certain standard denotations for
the Rewriting Logic variables occurring in the rules, possibly in indexed or
primed form. We let $e$ denote an \Erlang expression, $p$ denote a pattern, $X$ 
denote an \Erlang variable, $a$ denote an atom, $v$
denote a value, $f$ denote a function name, and $c$ denote a clause. Moreover
$\alpha$ refers to a transition label, $cs$ to a list of clauses,
$i,j,k$ to pids, $q$ to a mailbox, $\rho$ to an environment, and $s$ to a
concurrent process system.

\subsubsection{Expression--Level Rules}

The first rule describes the recursive evaluation of lists. Due to the
leftmost--innermost evaluation strategy of \Erlang we have to start the
evaluation with the first expression in the list constructor.
\[\inference[\rl{list$_1$}]{
    \process{\tau}{e_1}{i}{q}{\rho} 
    \trans{}
    \process{\alpha}{e_1'}{i}{q'}{\rho'} 
  }{ 
    \process{\tau}{\LIST{e_1 \BAR e_2}}{i}{q}{\rho} 
    \trans{}
    \process{\alpha}{\LIST{e_1' \BAR e_2}}{i}{q'}{\rho'} 
  } 
\] 

In general we assume in our formalization that a process--level rule is only
applicable if the latest transition label of the respective process was
$\tau$. Otherwise the label indicates a side effect which should be handled by
another appropriate (system--level) rule. 

As soon as the first subexpression of the list constructor is irreducible
(i.e., a value), the evaluation proceeds with the second subexpression:
\[\inference[\rl{list$_2$}]{
    \process{\tau}{e}{i}{q}{\rho}
    \trans{}
    \process{\alpha}{e'}{i}{q'}{\rho'}
  }{
    \process{\tau}{\LIST{v \BAR e}}{i}{q}{\rho}
    \trans{}
    \process{\alpha}{\LIST{v \BAR e'}}{i}{q'}{\rho'}
  }
\]

The following rules deal with pattern--matching operations. Here we just
formalize the \CASE construct. 
\[\inference[\rl{case$_1$}]{ 
    \process{\tau}{e}{i}{q}{\rho} 
    \trans{}
    \process{\alpha}{e'}{i}{q'}{\rho'} 
  }{
    \begin{array}{@{}r@{~}l@{}}
      ~\\[-5ex]
      & \process{\tau}{\CASEOF{e}{cs}}{i}{q}{\rho} \\[-1.5ex]
      \trans{}
      & \process{\alpha}{\CASEOF{e'}{cs}}{i}{b'}{\rho}
    \end{array}
  }
\]
\[\inference[\rl{case$_2$}]{
    \MATCHC{v}{cs}{\rho} = (e,\rho')
  }{ 
    \process{\tau}{\CASEOF{v}{cs}}{i}{q}{\rho}
    \trans{}
    \process{\tau}{e}{i}{q}{\rho'} 
  } 
\]

Finally we consider one of the \Erlang builtin functions which evoke side
effects on the system level of the semantics.
\[\inference[\rl{spawn}]{
    j = \NEWPID
  }{
      \process{\tau}{\SPAWN\ARGS{a \COMMA v}}{i}{q}{\rho}
      \trans{}
      \process{\SPN{a}{v}{j}}{j}{i}{q}{\rho}
  } 
\]
Here \NEWPID is a function returning a fresh pid which uniquely
identifies the new process, and which is returned as the result of the call of
\SPAWN.

The following rule handles one of the central concepts of \Erlang:
asynchronous sending of messages. As we shall see the message will be appended
to the mailbox of the target process. Note that a process can also send a
message to itself.
\[\inference[\rl{send}]{}{
    \process{\tau}{j \BANG v}{i}{q}{\rho} 
    \trans{}
    \process{\MSG{j}{v}}{v}{i}{q}{\rho} 
  } 
\]

\subsubsection{System--Level Rules}

The first rule just expresses that if a single process in a concurrent
system performs a computation step then so does the complete system.
\[\inference[\rl{Silent}]{
    \process{\tau}{e}{i}{q}{\rho}
    \trans{}
    \process{\tau}{e'}{i}{q'}{\rho'}
  }{
    \process{\tau}{e}{i}{q}{\rho} \parallel s
    \trans{}
    \process{\tau}{e'}{i}{q'}{\rho'} \parallel s
  }
\]

Process generation is formalized as follows. The \SPAWN builtin function comes
with two arguments: a function atom, and a list of arguments.
The new process will call this function with these arguments, starting with
the empty mailbox and the empty environment. 
\[\inference[\rl{Spawn}]{
    \process{\tau}{e}{i}{q}{\rho}
    \trans{}
    \process{\SPN{a}{v}{j}}{e'}{i}{q'}{\rho'}
  }{
    \begin{array}{@{}r@{~}l@{}}
      ~\\[-5ex]
      & \process{\tau}{e}{i}{q}{\rho} \parallel s \\[-1.5ex]
      \trans{}
      & \process{\tau}{e'}{i}{q'}{\rho'} \parallel s \parallel
        \process{\tau}{a \ARGS{v}}{j}{\NIL}{\NIL}
    \end{array}
  }
\]

Next we specify how a message is stored in the mailbox of the receiving
process. 
\[\inference[\rl{Com}]{
    \process{\tau}{e_1}{i}{q_1}{\rho_1}
    \trans{}
    \process{\MSG{j}{v}}{e_1'}{i}{q_1'}{\rho_1'}
  }{
    \begin{array}{@{}r@{~}l@{}}
      ~\\[-5ex]
      & \process{\tau}{e_1}{i}{q_1}{\rho_1} \parallel
        \process{\tau}{e_2}{j}{q_2}{\rho_2} \parallel s \\[-1.5ex]
      \trans{}
      & \process{\tau}{e_1'}{i}{q_1'}{\rho_1'} \parallel
        \process{\tau}{e_2}{j}{q_2 \cdot v}{\rho_2} \parallel s
    \end{array}
  }
\]

The next rule handles the situation when a process terminates normally, i.e.,
evaluates its expression to a value, becoming dead afterwards.
\[\inference[\rl{Termination}]{}{
    \process{\tau}{v}{i}{q}{\rho} \parallel s
    \trans{}
    \dead{i} \parallel s
  }
\]

Many more rules are required to complete the definition of our
rewrite theory $\ORT$ for
\Erlang. Together with an initial expression $e_0$ (and a
collection of modules with function definitions), it defines a labelled
transition system as follows.
\[T = (S, s_0, \trans{})\]
where $S$ is the set of states defined by
$S = \{s \mid s_0 \rtrans{} s\}$
and $s_0$ is the initial state given by
$s_0 = \process{\tau}{e_0}{i_0}{\NIL}{\NIL}$
for some initial pid $i_0$.

Note that the state space $S$ is infinite in general, due to several reasons:
\begin{itemize}
  \item \Erlang supports unbounded data structures, such as integers or lists.
  \item The unbounded use of recursive function calls can give rise to
    arbitrarily large expressions.
  \item The mailbox of a process can store an unbounded number of messages.
  \item Moreover the combination of dynamic process creation with
    recursive functions gives rise to infinite state spaces.
\end{itemize}
However employing the implementation of our semantics, which will be sket-ched
in the following, it can be shown that the state space of the
locker example from Section~\ref{SctErlang} is finite. It comprises
approximately 180 states. Experimenting with a varying number of client
processes shows the (expected) effect that the state space exponentially grows
with the number of clients---a typical example for the state--explosion problem
in model--checking applications. Approaches to alleviate this problem will be
discussed in Section~\ref{SctEqu}.

At first glance it might look surprising that the locker system possesses a
finite state space although both the \code{locker} and the \code{client}
function are recursively defined. This can be explained by the fact that only
\emph{tail recursion} is employed, that is, a recursive function call can
occur only as the last expression in the body of the respective function.
Hence such a call corresponds to a jump to the beginning of the respective
function body, and thus the size of the expression which represents the
control state of the computation is bounded. In fact tail recursion gives
rise to an implementation technique called \emph{last call optimization} which
allows to evaluate such functions in constant memory space (see \cite{Ste77} for the general idea and
\cite[Sct.~9.1]{AVWW96} for \Erlang--specific details).

\subsection{Implementation in \ELAN}

To develop an evaluator prototype for \Erlang we have chosen an 
existing implementation of the Rewriting
Logic framework, the \ELAN tool (cf.\ \cite{ELAN}), which is being developed
within the PROTHEO group at the LORIA research institute in Nancy, France.

The \ELAN system provides an environment for specifying and prototyping
deduction systems in a language based on rewrite rules whose application can
be controlled by strategies. \ELAN can be employed either as a logical
framework or to describe and execute both deterministic and non--deterministic
rule--based processes. Here we exploit the second feature by defining an
executable specification of the semantics of \Erlang.

System specifications are given in special \ELAN modules called \emph{element
modules}. In an element module one can import other modules and define the
sorts, operators, and rewrite rules of a rewrite theory. Equational reasoning
is supported by an efficient $\AC$--rewriting engine. Moreover it is possible
to describe \emph{strategies}, which define the way (i.e., the order and the
position) in which the rules can be applied to terms. These features proved
to be very useful for the implementation of our semantics for \Erlang. The
details can be found in \cite{Ami02}.

As seen before,
using this implementation it is possible, e.g., to compute the transition
system of the locker example from Section~\ref{SctErlang} with a varying
number of client processes (which are all spawned upon the initial call of the
\code{start} function). This enables us to show that the system indeed
exhibits the required properties (mutual exclusion etc.) which were mentioned
in the beginning of this chapter. We will come back to this issue in the 
following section.



\section{Equational Abstractions}
\label{SctEqu}

Note that so far $\ER$, the set of oriented equations, was only used to
implement the auxiliary functions occurring in the Rewriting Logic
specification of \Erlang. This is somehow in contradiction to the initial
motivation of introducing equations in order to reduce the number of rewriting
rules and/or the state space of the transition system.

In the following we sketch two approaches which target the second goal. The
first idea is to ``hide'' those computations in a given \Erlang program which
do not involve side effects, using the equational theory. In this way we
obtain an abstracted transition system which represents only those state
changes one would like to observe.

More concretely, we move those process--level rules which just produce $\tau$
transitions from $\TPrc$ to $\ER$, that is, we let
\[\begin{array}{rcl}
    \Ttau  & = & \left\{\inference[]{}{p \trans{} p'} \mid
                 p' \mbox{ is of the form } 
                 \process{\tau}{\ldots}{\ldots}{\ldots}{\ldots}
                 \right\} \\
           & = & \{\rl{list$_1$}, \rl{list$_2$}, \ldots\} \\
    \TPrc' & = & \TPrc \setminus \Ttau \\
    \ER'   & = & \ER \cup \Ttau
  \end{array}
\]

In this way, ``uninteresting'' computations such as the invocation of a
function by replacing the call with its body, or the evaluation of
a \CASE expression, do not unnecessarily increase the state space anymore but
are instead hidden in the current state.

Of course the question whether a certain action of the program is interesting
or not depends on the application. Therefore the choice of the set $\Ttau$ can
vary for different verification problems. In any case a prime requirement is
that the choice of the abstraction mapping ensures preservation of correctness
properties: \emph{false positives} should be excluded, that is, a property
checked to be true for the abstract system should also hold for the concrete
system being modelled. On the other hand, the term \emph{false negative}
refers to the (less critical) case that the abstract system exhibits an error
which cannot be traced back to the concrete system. Here the abstraction is
chosen too coarse, and the inspection of the error situation may then suggest
a way to refine it.

Figure~\ref{FigLockerLTS} shows the abstracted transition system of the locker
example with two clients, as discussed in Section~\ref{SctErlang}. Here
\key{start} denotes the initial state, consisting of a single process which
evaluates the function call \code{start}(). The locker process
is denoted by $L$, possibly superscripted by client numbers (1 and/or 2) to
indicate the contents of its message mailbox. For example, $L^{12}$ represents
a locker process which has stored \code{request} messages from the first and
the second client (in that order) in its mailbox. The clients are denoted by
$C_1$ and $C_2$ where a bar indicates that the respective client is in the
critical section.

The transitions are labelled to indicate the type of action which caused the
change of the state. Here \key{spn} refers to the creation of a process,
$\key{req}_i$ and $\key{rel}_i$ indicate that $C_i$ sends a \code{request}
or, respectively, a \code{release} message to the locker (where $i \in \{1,
2\}$), and $\key{ok}_i$ denotes the admission from the locker to enter the
critical section.

\fig{Abstracted Locker Transition System}{FigLockerLTS}{\footnotesize\centering
  $ %\scriptstyle
    \begin{psmatrix}[rowsep=1cm,colsep=0.25cm]
      & 
      & 
      &
      & \key{start}    
      & 
      & \\
      & 
      & 
      &
      & L
      & 
      & \\
      & 
      & 
      &       
      & L \parallel C_1 
      & 
      & \\
      & 
      & 
      & L \parallel C_1 \parallel C_2 
      &
      & L^1 \parallel C_1 
      & \\ 
      & L^2 \parallel C_1 \parallel C_2 
      &
      &
      & L^1 \parallel C_1 \parallel C_2
      &
      & L \parallel \compl{C_1} \\ 
      L^{21} \parallel C_1 \parallel C_2 
      &
      & L \parallel C_1 \parallel \compl{C_2}
      & L^{12} \parallel C_1 \parallel C_2 
      &
      & L \parallel \compl{C_1} \parallel C_2
      & \\ 
      & L^1 \parallel C_1 \parallel \compl{C_2}
      & 
      & 
      & L^2 \parallel \compl{C_1} \parallel C_2 
      &
      & 
      \psset{nodesep=2pt,labelsep=2pt,arrows=->}
      \ncline{1,5}{2,5}>{\key{spn}}
      \ncline{2,5}{3,5}>{\key{spn}}
      \ncline{3,5}{4,4}>{\key{spn}}
      \ncline{3,5}{4,6}<{\key{req}_1}
      \ncline{4,4}{5,2}>{\key{req}_2}
      \ncline{4,4}{5,5}>{\key{req}_1}
      \ncline{4,6}{5,5}>{\key{spn}}
      \ncline{4,6}{5,7}<{\key{ok}_1}
      \ncline{5,2}{6,1}>{\key{req}_1}
      \ncline{5,2}{6,3}>{\key{ok}_2}
      \ncline{5,5}{6,4}<{\key{req}_2}
      \ncline{5,5}{6,6}>{\key{ok}_1}
      \ncline{5,7}{6,6}<{\key{spn}}
      \ncarc[arcangle=-30]{5,7}{3,5}^[npos=.7]{\key{rel}_1}
      \ncline{6,1}{7,2}>{\key{ok}_2}
      \ncline{6,3}{7,2}<{\key{req}_1}
      \ncline{6,3}{4,4}>{\key{rel}_2}
      \ncline{6,4}{7,5}>{\key{ok}_1}
      \ncline{6,6}{7,5}<{\key{req}_2}
      \ncarc[arcangle=30,border=2pt]{6,6}{4,4}<{\key{rel}_1}
      \ncarc[arcangle=-40,border=2pt]{7,2}{5,5}>{\key{rel}_2}
      \ncarc[arcangle=40,border=2pt]{7,5}{5,2}<{\key{rel}_1}
    \end{psmatrix}
  $
  \bigskip
}

Thus the abstracted system contains 14 states while the standard
transition system derived in the original semantics comprises 182 states.
Table~\ref{TabLockerLTS} shows, for several numbers of clients, both the size
of the state space and the time required to compute it. The results are very
promising, inviting to further investigate the benefits of equational
abstractions for \Erlang programs.

\tab{Original vs.\ Abstracted Locker Systems}{TabLockerLTS}{%
  \bigskip
  \begin{tabular}{|l||l|l|l|}\hline
    Locker with    & 1 client  & 2 clients  & 3 clients \\\hline\hline
    Original LTS   & 65 states & 182 states & 536 states \\
                   & 10 min    & 55 min     & 380 min \\\hline
    Abstracted LTS & 5 states  & 14 states  & 42 states \\
                   & 20 sec    & 90 sec     & 13 min \\\hline
  \end{tabular}
}

\bigskip
As mentioned earlier, using the formal semantics we can derive that our
implementation of the locker behaves correctly. More concretely, the
(abstracted) transition system from Figure~\ref{FigLockerLTS} possesses the
following properties: 
\begin{description} 
  \item[no deadlock:] there exists no cyclic chain of processes waiting for
    each other to continue, which can be deduced from the fact that every
    state in the system has a direct successor,
  \item[mutual exclusion:] no two clients can gain access to the resource at
    the same time since in every state there is at most one client process in
    the critical section (i.e., marked with a bar), and 
  \item[no starvation:] every client enabled to enter the critical section
    eventually gets its demanded access. At first glance this property seems 
    to be violated since there exists a (reachable) cycle in the system in
    which the second client is never active: 
    \[\key{start} 
      \trans{\key{spn}} L
      \trans{\key{spn}} L \parallel C_1 
      \trans{\key{req}_1} L^1 \parallel C_1
      \trans{\key{ok}_1} L \parallel \compl{C_1} 
      \trans{\key{rel}_1} L \parallel C_1 
      \trans{\key{req}_1} \ldots 
    \] 
    (A similar scenario can be constructed to show that also the first client
    could starve.) 

    However this cycle indefinitely excludes the \code{start} function in the
    main process from spawning the second client process, and is therefore in
    contradiction to the requirements which are postulated for every
    implementation of the \Erlang runtime system: any process which is enabled
    for execution will eventually be run, using a bounded time slice. Thus we
    can indeed guarantee that $C_2$ will be spawned and, moreover, that
    $C_2$ will obtain the chance to send a \code{request} message to the
    locker. In other words, the system will eventually reach a state in
    which the index $2$ appears in the mailbox superscript of the locker
    process. The same applies to the first client.

    To establish the no--starvation property it therefore suffices to show
    that, for each $i \in \{1, 2\}$, from every state in which the locker
    is superscripted by $i$, each possible continuation will
    eventually pass through an action of the form $\key{ok}_i$. This is easily
    verified by inspecting the transition system.
\end{description}

Of course it is possible to have these properties automatically be checked by
suitable tools such as \Truth.

\bigskip
Let us now turn towards the second approach which employs directed equations
to define abstraction mappings. The following example shows that such
equations can sometimes be used to reduce an infinite to a finite system. The
code fragment given below implements a simple concurrent server which repeatedly
accepts an incoming query in the form of a triple which is tagged by the
atom \code{request}, and which contains the request itself (matched by
the variable \code{Request}) and the pid of the client process
(\code{Client}). It then spawns a process which serves the request by invoking
the \code{handle} function (which is not shown here), and by sending the
result back to the client as a tuple tagged by the \code{response} atom.

\medskip
\begin{lstlisting}[gobble=2]{}
  concurrent_server() ->
    receive
      {request, Request, Client} ->
        spawn(serve, [Request, Client])
    end,
    concurrent_server(). 

  serve(Request, Client) -> 
    Client!{response, handle(Request)}.
\end{lstlisting} 

Having server requests handled in this way, i.e., by separate concurrent
subprocesses, represents a programming technique which is widely used in
\Erlang applications. Apart from the fact that the load of the host
running the server process can easily be balanced by controlling the number of
concurrent processes, it offers the great advantage that the computations of
the server itself and of the single requests are kept in isolation, and cannot
influence each other therefore (cf.\ \cite[Chapter~8]{AVWW96}). However it has
the consequence that after every completion of a serving process a dead process
remains in the system. Hence if the total number of calls to the concurrent
server is not bounded, the number of system states is not bounded either.

Using the reduction
rule $s \parallel \dead{i} \trans{} s$ (where $s$ denotes a system and
$i$ a pid) it is possible to obtain a finite transition system, assuming that
the number of simultaneous calls to the server is bounded, and that the
processing of a single request involves only finite behavior.



\section{Conclusions}
\label{SctConcl}

In this paper we have proposed a variant of Meseguer's Rewriting Logic as a
semantic framework in which the operational semantics of the \Erlang programming 
language can be formalized. In particular we have seen that the equational 
reasoning
allows to test and to compare different abstractions to reduce the size of the
state space. Thus it avoids the error--prone and slow construction of abstract
system models by hand.

The concrete runtime figures have shown that a lot remains to be done with
respect to the optimization of both the tools implementing the Rewriting Logic
framework (\ELAN in our case) and the description of the programming language
under consideration, which usually involves rather complex semantic structures.

Of course one should not be over--optimistic in that respect. There will
always be a certain complexity implied by conditional equational term
rewriting modulo associativity and commutativity, which is the essential basis
of our framework. It is therefore illusory to expect that an evaluator which
has been automatically derived from the Rewriting Logic definition of a
specification language will achieve the efficiency of a hand--written,
optimized implementation. Instead the ``executable specification'' approach
proves its strength with respect to flexibility and user--friendliness. Thus
\ELAN and similar systems should be regarded as rapid--prototyping tools for
frontends of verification tools. This aspect is being discussed in \cite{LN99}.

Further techniques for formally abstracting \Erlang programs with infinite
state spaces to finite--state form, followed by fully automated model
checking, are investigated by F.~Huch in \cite{Huc99}. He employs the
technique of \emph{abstract interpretation} to define finite--domain
abstractions for unbounded data structures. This enables him to show that the
abstract interpretation of an \Erlang program in which only tail recursion is
employed, which spawns only a finite number of processes, and which uses only
finite parts of the mailboxes induces a finite transition system, and is thus
amenable to classical model--checking methods.

Since an infinite state space can not always be reduced to a finite one
without losing ``essential'' information, it is natural to employ interactive
\emph{theorem--proving assistants} such as the \EVT \Erlang Verification Tool
(see \cite{FGN01,NFG01,FGN02}) to establish the desired system properties
in such situations. In any case the semantics of \Erlang has to be
implemented, mapping \Erlang programs to transition systems. (Additionally,
the abstraction functions have to be implemented for the model--checking
approach.) Here again our compiler--generating approach can be used for
automatically deriving the verification tool frontend.



\bibliographystyle{plain}
\bibliography{abbreviations,mcs,noll}
\end{document}
